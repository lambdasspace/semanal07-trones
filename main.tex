\documentclass{article}
\usepackage{graphicx} % Required for inserting images

\usepackage{listings}
\usepackage{color}

\definecolor{dkgreen}{rgb}{0,0.6,0}
\definecolor{gray}{rgb}{0.5,0.5,0.5}
\definecolor{mauve}{rgb}{0.58,0,0.82}

\lstset{frame=tb,
  language=Haskell,
  aboveskip=3mm,
  belowskip=3mm,
  showstringspaces=false,
  columns=flexible,
  basicstyle={\small\ttfamily},
  numbers=none,
  numberstyle=\tiny\color{gray},
  keywordstyle=\color{blue},
  commentstyle=\color{dkgreen},
  stringstyle=\color{mauve},
  breaklines=true,
  breakatwhitespace=true,
  tabsize=3
}

\title{Plantilla}
\author{Fernando Romero Cruz}
\date{Octubre 2024}

\begin{document}

\maketitle

\section*{Ejercicio 3}

Definición de la función recursiva \texttt{ocurrenciasElementos}:

\begin{lstlisting}
-- Implementacion convencional
ocurrenciasElementos :: (Eq a) => [a] -> [a] -> [(a,Int)]
ocurrenciasElementos lista [] = []
ocurrenciasElementos lista (x:xs) = (x,ocurrencias) : rec
    where ocurrencias = length $ filter (== x) lista
           rec = ocurrenciasElementos lista (filter (/= x) xs)
\end{lstlisting}

Registros de activación de la función anterior con la llamada:
\begin{verbatim}
    ocurrenciasElementos [1,2,3] [1,2]
\end{verbatim}

Nombre: \texttt{ocurrenciasElementos}\\
Parametro: \texttt{[1,2,3] [1,2]}\\

\end{document}