\documentclass{article}
\usepackage{graphicx} % Required for inserting images
\usepackage{amsmath}

\usepackage{listings}
\usepackage{color}

\hbadness=99999  % or any number >=10000

\definecolor{dkgreen}{rgb}{0,0.6,0}
\definecolor{gray}{rgb}{0.5,0.5,0.5}
\definecolor{mauve}{rgb}{0.58,0,0.82}

\lstset{frame=tb,
  language=Haskell,
  aboveskip=3mm,
  belowskip=3mm,
  showstringspaces=false,
  columns=flexible,
  basicstyle={\small\ttfamily},
  numbers=none,
  numberstyle=\tiny\color{gray},
  keywordstyle=\color{blue},
  commentstyle=\color{dkgreen},
  stringstyle=\color{mauve},
  breaklines=true,
  breakatwhitespace=true,
  tabsize=3
}

\title{Semanal 7}
\author{Diego Castro Rendón\\ Márquez Corona Danna Lizette\\ Fernando Romero Cruz}
\date{Octubre 2024}

\begin{document}

\maketitle

\section*{Ejercicio 1}

La expresión:

\begin{verbatim}
(let (sum (lambda (n) (if0 n 0 (+ n (sum (- n 1))))))
   (sum 5))
\end{verbatim}

\begin{enumerate}
    \item \textbf{Expresión:} 
    Usa \texttt{let} para poder definir \texttt{sum}, una función recursiva que suma los números desde \( n \) hasta \( 0 \). 
    La función utiliza \texttt{if0}, que va a verificar si \( n \) es igual a \( 0 \). Si sí lo es, entonces va a  devolver \( 0 \) de lo contrario, va a devolver \( n \) más la suma de los números desde \( n-1 \).

    \item \textbf{Ejecución:}
    Se llama a \( \texttt{(sum 5)} \) pero esto genera un error de variable libre porque \texttt{sum} no está definido dentro del cuerpo de la función lambda. 
    La función intenta llamar a \texttt{sum} recursivamente, pero no puede encontrar \texttt{sum} en su entorno, ya que no tiene acceso a su propia definición.
\end{enumerate}

Modificación usando el Combinador de Punto Fijo \( Y \)

Ahora, modificamos la expresión utilizando el Combinador de Punto Fijo \( Y \):

\begin{verbatim}
(let (Y (lambda (f) ((lambda (x) (f (x x))) (lambda (x) (f (x x))))))
      (sum (Y (lambda (sum) (lambda (n) (if0 n 0 (+ n (sum (- n 1)))))))))
   (sum 5))
\end{verbatim}

\begin{enumerate}
    \item \textbf{Modificación:}
    Vamos a definir \( Y \) como el Combinador de Punto Fijo esto porque nos permite que las funciones se llamen a sí mismas.
    \texttt{sum} se define como la aplicación de \( Y \) a una función que va a tomar como argumento \texttt{sum}. Entonces ahora dentro de la definición de \texttt{sum}, podemos llamar a \texttt{sum} sin que generemos un error de variable libre.

    \item \textbf{Ejecución:}
    Cuando llamamos a \( \texttt{(sum 5)} \), ahora \texttt{sum} está definido correctamente.
    La ejecución va a ser de la siguiente manera:
    \begin{align*}
        \text{sum(5)} & = 5 + \text{sum(4)} \\
        \text{sum(4)} & = 4 + \text{sum(3)} \\
        \text{sum(3)} & = 3 + \text{sum(2)} \\
        \text{sum(2)} & = 2 + \text{sum(1)} \\
        \text{sum(1)} & = 1 + \text{sum(0)} \\
        \text{sum(0)} & = 0 \\
    \end{align*}
    Esto se evalúa a \( 5 + 4 + 3 + 2 + 1 + 0 = 15 \).
\end{enumerate}

\begin{enumerate}
    \item \textbf{Primera expresión:} Va a generar un error de variable libre porque \texttt{sum} no tiene acceso a su propia definición.
    \item \textbf{Segunda expresión:} Funciona bien y el resultado de \( \texttt{(sum 5)} \) es \( 15 \), que es la suma de los números del 1 al 5.
\end{enumerate}

\section{Ejercicio 2}

Evaluar la siguiente expresión en Racket, explicar su resultado y dar la continuación asociada a evaluar usando la notación $\lambda \uparrow$:

\begin{verbatim}
> (define c #f)
> (+ 1 (+ 2 (+ 3 (+ (let/cc k (set! c k) 4) 5))))
> (c 10)
\end{verbatim}

En la primera línea solamente definimos $c$ como falso. Luego, en la segunda empezamos con la suma, donde la parte más anidada, con el \texttt{let/cc} captura la continuación actual en la variable $k$, la cual se almacena en $c$. Después de esto, se da el valor $4$, así que (+ (\texttt{let/cc} \, k \, (\texttt{set!} \, c \, k) \, 4) \, 5) se evalúa como  (+ 4 \, 5) = 9.
(+ 3 \, 9) se evalúa como $12$.\\
(+ 2 \, 12) se evalúa como $14$.\\
(+ 1 \, 14) se evalúa como $15$.\\
Por lo tanto, el valor de la segunda línea es $15$.\\
\newline
Luego, en la última línea invocamos la continuación almacenada en $c$ que vimos en el paso anterior. Aquí, (c \, 10)  reemplaza la parte donde se evaluaba (+ 4 \, 5) (donde está el \texttt{set!}) por (+ 10 \, 5), que da $15$. Después, la evaluación sigue como antes:
(+ 3 \, 15) = 18, \quad (+ 2 \, 18) = 20, \quad \text{y finalmente} \quad (+ 1 \, 20) = \textbf{21}.\\

\textbf{Notación $\lambda \uparrow$}: $\lambda(v)=$\texttt{(+ 1 (+ 2 (+ 3 (+ v 5))))}\\

\section*{Ejercicio 3}

\subsection*{Definición de la función recursiva \texttt{ocurrenciasElementos}:}

\begin{lstlisting}
-- Implementacion convencional
ocurrenciasElementos :: (Eq a) => [a] -> [a] -> [(a,Int)]
ocurrenciasElementos lista [] = []
ocurrenciasElementos lista (x:xs) = (x,ocurrencias) : rec
    where ocurrencias = length $ filter (== x) lista
           rec = ocurrenciasElementos lista (filter (/= x) xs)
\end{lstlisting}

\subsection*{Registros de activación}
Sea la llamada: \texttt{ocurrenciasElementos [1,2,3] [1,2]}\\
\newline
Los registros de activación son:\\
\newline
Nombre: \texttt{ocurrenciasElementos}\\
Parámetro: \texttt{[1,2,3] [1,2]}\\
Nivel: 0\\
Valor de Retorno: \texttt{(1,1):ocurrenciasElementos [1,2,3] [2]}\\
\newline
Nombre: \texttt{ocurrenciasElementos}\\
Parámetro: \texttt{[1,2,3] [2]}\\
Nivel: 1\\
Valor de Retorno: \texttt{(2,1):ocurrenciasElementos [1,2,3] []}\\
\newline
Nombre: \texttt{ocurrenciasElementos}\\
Parametro: \texttt{[1,2,3] []}\\
Nivel:2\\
Valor de retorno: \texttt{[]}\\
\newline
\subsection*{Nueva Función}

\begin{lstlisting}
  -- Funcion redefinida con recursion de cola
  ocurRecurCola :: (Eq a) => [a] -> [a] -> [(a,Int)]
  ocurRecurCola lista nums = aux lista (reverse nums) []
    where aux :: (Eq a) => [a] -> [a] -> [(a,Int)] -> [(a,Int)]
           aux lista [] acc = acc
           aux lista (x:xs) acc = aux lista filt (elem:acc)
             where filt = filter (/= x) xs
                    elem = (x,length (filter (== x) lista))
\end{lstlisting}

\subsection*{Registros de activación}
Sea la llamada: \texttt{ocurRecurCola [1,2,3] [1,2]}\\
\newline
Los registros de activación son:\\
\newline
Nombre: \texttt{ocurRecurCola}\\
Parámetro: \texttt{[1,2,3] [1,2]}\\
Nivel: 0\\
Valor de Retorno: \texttt{aux [1,2,3] [2,1] []}\\
\newline
Nombre: \texttt{aux}\\
Parámetro: \texttt{[1,2,3] [2,1] []}\\
Nivel: 0\\
Valor de Retorno: \texttt{aux [1,2,3] [1] [(2,1)]}\\
\newline
Nombre: \texttt{aux}\\
Parametro: \texttt{[1,2,3] [1] [(2,1)]}\\
Nivel: 0\\
Valor de retorno: \texttt{[aux [1,2,3] [] [(1,1),(2,1)]]}\\
\newline
Nombre: \texttt{aux}\\
Parámetro: \texttt{[1,2,3] [] [(1,1),(2,1)]}\\
Nivel: 0\\
Valor de retorno: \texttt{[(1,1),(2,1)]}\\

\end{document}